\part{HIC Devices} \label{par:hic_devices} \index{HIC devices|(}

\section*{What are the ``HIC Devices''?}

High Index Contrast (HIC)\index{HIC devices} dielectric waveguides are
optical devices in which the refractive indices of core and cladding
differ greatly, much more than in conventional optical fibers. HIC
devices exhibit, for this reason, highly confined optical modes and
allow for waveguides to be spaced closely together without inducing
crosstalk and the propagating field to be guided around sharp bends
with minimal radiative loss \cite{watts_integrated}. This goes in the
same direction as the need of integration of more and more
functionalities inside optical chips, their grown in complexity and
the will to reduce prices \cite{exp_pirelli}.

However, as the index contrast is increased, the differences between
the lateral boundary conditions for TE and TM modes become more
pronounced, causing critical design parameters, such as propagation
constants, coupling coefficients and scattering losses to be
polarization dependent\index{Polarization!dependence}.

To allow polarization independent performance, a necessary feature for
a standard single-mode-fiber-based communication link, the
polarization sensitivity can be circumvented by the implementation of
a polarization diversity\index{Polarization!diversity} scheme
\cite{madsen_optical}. Such an approach requires the input
polarization, from the optical fiber, to be split into two orthogonal
components. Then, rotation of one of them allows one single
polarization to be realized on chip, on two parallel paths of
identical devices. A possible scheme is reported in
\figref{fig:exp_fig1}.

\begin{figure}[htbp]
  \begin{center}
    \resizebox{12cm}{!}{\input{pics/exp_fig1.pdf_t}}
  \end{center}
  \caption{Polarization diversity\index{Polarization!diversity|fig}
    scheme. The function $F(\omega)$ performs some kind of operations
    on a polarized input signal. The splitter and the rotator before
    it ensure that the input polarization is the one the function was
    designed to work with. A second rotator is needed in the
    recombination stage. Note that the two rotation stages are split
    on the two arms of the scheme, to make their side effects symmetric
    (losses, crosstalk, etc.).}
  \label{fig:exp_fig1}
\end{figure}

In the next chapter, the study of a polarization rotator (the block in
red in \figref{fig:exp_fig1}) is reported.

\input polrot

\index{HIC devices|)}