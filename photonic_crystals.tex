\chapter{Photonic Crystals}

\begin{enumerate}
\item
  introduction: what are they -> see my thesis
\item
  XXX how do you describe them: crystal lattices, reciprocal lattice
\item
  how do you study them analitycally: yeh yariv electromagnetic -> theory
\item
  what can you do with them:
  \begin{enumerate}
    \item
      broeng photonic, maradudin out of plane -> photonic crystal
      fibers
    \item
      boscolo superprism, enoch numerical -> superprism -> NIM
  \end{enumerate}
\end{enumerate}

\section{Crystal Lattices}

Periodic structures like photonic crystals can be matematically
described by the concept of \emph{Bravais lattice}. A \emph{Bravais
  lattice} is an infinite array of discrete points with an
\emph{arrangement} and an \emph{orientation} that appear
\emph{exactly} the same from whichever of the points the array is
viewed \OKKIO{citare kittel solid state physics}. In more rigorous
words, for a three dimensional lattice, it consists of all the points
with position vectors $\Vector{R} = n_1 \Vector{R_1} + n_2
\Vector{R_2} + n_3 \Vector{R_3}$, with $n_i$, $i = 1,2,3$ integer
number. The vectors $\Vector{R_i}$, $i = 1,2,3$, are called
\emph{primitive vectors} and must be linearly indipendent (non
coplanar).

For a given Bravais lattice, the choice of the primitive vectors is
not unique: indeed, there are infinite valid choices!

\OKKIO{figura di un lattice 2D e 3D. fig. 4.4}

Since all points are equivalent, the Bravais lattice is infinite in
extent. This is clearly an approximation in the description of a real
crystal, which is never infinite: if it's large enough, though, the
vast majority of the points will be too far from the boundaries to be
affected by them and the description will be accurate.

The volume of space which, when translated through all the vectors in
the Bravais lattice, just fills all of the space without overlapping
or leaving voids is called \emph{primitive cell}. Again, there is no
unique choice of the primitive cell for a given Bravais lattice, but,
as long as every primitive cell must contain exactly one lattice point
(unless it is so positioned that there are points on its surface) and
the density of points in the lattice is constant, the volume of the
primitive cell is constant for each possible choice.

\OKKIO{fig. 4.10 kittel}

The most frequently used primitive cell, given a Bravais lattice, is
the \emph{Wigner-Seitz} primitive cell. It is defined as the region of
space which is closer to a point of the lattice than to every other
point. Its construction resembles closely the costruction of the
Vorono\"i diagram of a Delaunay mesh \ref{subsec:delaunay_voronoi}.

\OKKIO{fig. 4.14}

To describe a real crystal both the description of the underlying
Bravais lattice and of the arrangements of atoms inside each unit cell
are needed. The latter is called \emph{basis} and a crystal is
sometimes referred to as a \emph{lattice with a basis}. For example, a
crystal with a honeycomb arrangement of atoms is not a Bravais lattice
if the unit cell is contains just one atom (the orientation uniformity
is missing), but it is if the unit cell is made of two atoms.

\OKKIO{fig. 4.17}

Given a Bravais lattice, another important concept is its
\emph{reciprocal lattice}, defined as the set of all the wave vectors
$\Vector{G}$ that yield plane waves with the same periodicity of the
lattice. Analitically, $\Vector{G}$ belongs to the reciprocal lattice
of a Bravais lattice of points $\Vector{R}$ if:
\begin{equation} \label{eqn:reciprocal_long}
  e^{\imath \DotProd{\Vector{G}}{(\Vector{r}+\Vector{R})}} = e^{\imath
  \DotProd{\Vector{G}}{\Vector{r}}}
\end{equation}
for any $\Vector{r}$ and for all $\Vector{R}$ in the Bravais
lattice. \ref{eqn:reciprocal_long} can be rewritten factoring out
$e^{\imath \DotProd{\Vector{G}}{\Vector{r}}}$, to obtain:
\begin{equation} \label{eqn:reciprocal}
  e^{\imath \DotProd{\Vector{G}}{\Vector{R}}} = 1
\end{equation}
for all $\Vector{R}$ in the Bravais lattice.

It's worth noting that the reciprocal lattice of a Bravais lattice is
itself a Bravais lattice. To verify it, just note that the reciprocal
lattice can be generated by the primitive vectors $\Vector{G_i}$, $i =
1,2,3$ (in three dimensions):
\begin{eqnarray*}
\Vector{G}_1 & = & 2 \pi
\frac{\CrossProd{\Vector{R}_2}{\Vector{R}_3}}{\DotProd{\Vector{R}_1}{(\CrossProd{\Vector{R}_2}{\Vector{R}_3})}} \\
\Vector{G}_1 & = & 2 \pi
\frac{\CrossProd{\Vector{R}_3}{\Vector{R}_1}}{\DotProd{\Vector{R}_1}{(\CrossProd{\Vector{R}_2}{\Vector{R}_3})}} \\
\Vector{G}_1 & = & 2 \pi
\frac{\CrossProd{\Vector{R}_1}{\Vector{R}_2}}{\DotProd{\Vector{R}_1}{(\CrossProd{\Vector{R}_2}{\Vector{R}_3})}}
\end{eqnarray*}
and that they satisfy:
$$
\DotProd{\Vector{G}_i}{\Vector{R}_j} = 2 \pi \delta_{ij}
$$
Now, any vector in the reciprocal lattice space $\Vector{k}$ can be
written as a linear combination of $\Vector{G}_i$, $i = 1,2,3$,
$\Vector{k} = k_1 \Vector{G}_1 + k_2 \Vector{G}_2 + k_3 \Vector{G}_3$
and any vector in the Bravais lattice space $\Vector{R}$ can be
written as $\Vector{R} = n_1 \Vector{R}_1 + n_2 \Vector{R}_2 + n_3
\Vector{R}_3$ with $n_i \in \AllNaturalSet$. Therefore:
$\DotProd{\Vector{k}}{\Vector{R}} = 2 \pi (k_1 n_1 + k_2 n_2 + k_3
n_3)$ and for $e^{\imath \DotProd{\Vector{k}}{\Vector{R}}}$ to be
unity for all $\Vector{R}$, $\DotProd{\Vector{k}}{\Vector{R}}$ must be
$2 \pi$ times an integer number for each choice of the integers
$n_i$. This requires that $k_i$ are integers as well. This proves that
the reciprocal lattice is a Bravais lattice.

From this consideration, we can define all the entities, defined for
Bravais lattices, on the reciprocal lattice.
\begin{itemize}
\item
  The reciprocal lattice of a reciprocal lattice turns out to be the original
  Bravais lattice.
\item
  The Wigner-Seitz primitive cell of a reciprocal lattice is called
  \emph{First Brillouin Zone} and it plays a critical role in the
  study of crystals.
\end{itemize}



Simulations are mainly done via the \emph{Plane Wave Expansion}
(\Code{PWE}) method.
