\part{Propagators} \label{par:propagators} \index{Propagators|(}

\section*{What are the ``Propagators''?}

Electromagnetic phenomena are fully described by Maxwell
equations. As any differential equations, they can be
mathematically solved if initial conditions and boundary conditions
are defined. We call ``propagators'' the techniques to solve Maxwell
equations, both in the time- and frequency-domain, once a domain, with
some boundaries, and some sources have been defined.

\input discretization_schemes

\input material_equations

\input time_domain

\input frequency_domain

\chapter*{Conclusions}

The present algorithm represents a novel method to discretize and
solve Maxwell equations. It shares some similarities with the
\emph{Finite Integration Technique} and the \emph{Finite Element}
method \cite{bossavit_yee}, but it's original in its integral approach
to describe and discretize the problem.

Simulations have shown that it is an effective way to solve
electromagnetic problems, both in the time- and
frequency-domain. Benchmarks presented in section
\ref{sec:validation_01} and followings show a good accuracy of the
results, if compared with commercial software.

On the other hand, the Author is convinced that the present method is
very effective and superior to other simpler algorithms like \FDTD or
\FEM only in two-dimensional simulation of frequency-domain problems. In the
time-domain, the advantage of having an unstructured grid, which seems
to be the only clear advantage over a conventional \FDTD algorithm, does
not pay the increased complexity of the algorithm, which directly
effects the possibility of an efficient implementation on a
computer. For three-dimensional frequency-domain problems, memory requirements are
still too stringent to be applied for very complex problems. Other
techniques must be used, that makes more assumptions on the problems
in order to simplify it. This is the case, for example, of the
\emph{Multiple Scattering}\index{Multiple Scattering method}
method \cite{tayeb_rigorous}, which, making the hypothesis of perfectly
circular scatterers, reduces the problem to an expansion over a
efficient basis functions set and translate it into an eigenvalue
problem, or the \emph{Eigenmode Expansion}\index{Eigenmode Expansion} technique, used in FIMMPROP
\cite{fimmprop}.

A two-dimensional frequency-domain implementation of the algorithm,
implemented by the Author, is commercially available from Photon
Design\index{Photon Design} \cite{photond} \index{Photon Design}.

\index{Propagators|)}

