\begin{preface}

  %% \Cite{Computational electromagnetism begins where electromagnetic
  %%   theory stops, and stops where engineering takes over.}{Alain
  %%   Bossavit}
  
  \Cite{Before solving a problem, we must first of all set it properly. We
    have a physical situation on the one hand, with a description
    (dimensions, values of physical parameters) and a query about this
    situation, coming from some interested party. The task of our party
    (the would-be Mathematical Modeler, Computer Scientist and Expert in
    the Manipulation of Electromagnetic Software Systems) is to formulate
    a relevant mathematical problem, liable to approximate solution
    (usually with a computer), and this solution should be in such final
    form that the query is answered, possibly with some error or
    uncertainty, but within a controlled and predictable
    margin. Mathematical modelling is the process by which such a
    correspondence between a physical situation and a mathematical problem
    is established.}{Alain Bossavit in \cite[page 32]{bossavit_computational}}

  The present thesis describes the mathematical models and their computer
  implementation to study and solve the electromagnetic problems faced in the
  three years of my Ph.D..

  In \parref{par:propagators}, a novel algorithm to study the
  propagation of light will be investigated, both in time- and in
  frequency-domain. Comparisons with commercial software will be worked
  out and conclusions will follow.

  In \parref{par:mode_solvers}, two algorithms, to study straight
  dielectric waveguides and periodic structures, will be
  discussed. Validations and comparisons with commercial software and
  literature will be presented.

  Finally, in \parref{par:hic_devices}, these algorithms will be applied
  to the study of a physical device, its design and its experimental
  validation.

  Invaluable source of ideas and suggestions have been all my collegues
  from the PICCO Project, who supported my interests on novel
  time-domain algorithms to study photonic crystals, from Photon Design,
  whose experience and friendship have made learning from them a joy,
  and from the FUNFOX Project, who allowed me to experiment my theorical
  studies. In every moment of these experiences, Professor Michele
  Midrio has been a constant reference and an enthusiastic supporter.

  \Sign{Lorenzo Bolla}{January 31, 2006}

\end{preface}

\begin{prefacetwo}

  \vspace{1cm}

  \Cite{Prima di risolver un problema, dobbiamo prima di tutto porlo nel
    modo corretto. Si parte da una situazione fisica, con una
    descrizione (attraverso valori e dimensioni di certi parametri fisici) e
    una domanda su tale situazione da parte di qualcuno. L'obiettivo del
    nostro gruppo (l'aspirante Matematico Applicato, Programmatore ed
    Esperto in Software per l'Elettromagnetismo) \`e di formulare un
    appropriato problema matematico, dotato di una soluzione
    approssimata (di solito ricavabile con un computer) tale da
    rispondere alla domanda posta, anche se con un certo margine di
    errore o incertezza, ma all'interno di margini controllati e
    predicibili. La creazione di modelli matematici \`e il processo
    attraverso il quale si realizza questa corrispondenza tra situazione
    fisica e problema matematico.}{Alain Bossavit in \cite[page
      32]{bossavit_computational} -- traduzione dell'Autore}

  La presente tesi descrive i modelli matematici e la relativa
  implementazione al calcolatore per lo studio di problemi di
  elettromagnetismo, affrontati nei tre anni del mio dottorato di
  ricerca.

  Nella Parte \ref{par:propagators}, \`e descritto un originale
  algoritmo per lo studio della propagazione della luce, sia nel dominio
  del tempo che della frequenza, e validato attraverso confronto con
  software commerciale.

  Nella Parte \ref{par:mode_solvers}, sono discussi due algoritmi: uno
  per lo studio di guide d'onda dielettriche a sezione costante, l'altro
  per lo studio di strutture dielettriche periodiche. Anche in questo
  caso, sono presentati validazione ed esempi tratti dalla letteratura e
  da software commerciale.

  Infine, nella Parte \ref{par:hic_devices}, questi algoritmi saranno
  applicati allo studio, al design e alla validazione sperimentale di un
  dispositivo reale.

  Un'inesauribile fonte di idee e suggerimenti sono stati tutti i miei
  colleghi del progetto europeo PICCO, per il supporto e l'incitamento
  nello studio di nuovi algoritmi dedicati alla propagazione di luce in
  cristalli fotonici, della Photon Design, la cui esperienza ed amicizia
  ha reso imparare da loro una gioia, e del progetto europeo FUNFOX, per
  l'op\-por\-tu\-ni\-t\`a di applicare praticamente i miei studi teorici. In
  ogni fase di queste esperienze, il Professor Michele Midrio ha
  rappresentato una presenza costante e una continua fonte di entusiasmo
  e incitamento.

  \Sign{Lorenzo Bolla}{31 Gennaio 2006}

\end{prefacetwo}
