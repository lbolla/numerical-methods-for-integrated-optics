\chapter{Discretization Schemes}
from the Maxwell's equations to a matricial formulation
\begin{itemize}
\item
  mapping of geometrical entities with physical quantities
  \cite[pag. 40]{bossavit_computational}: constituive laws are
  mappings between p-forms and (3-p)-forms.
\item
  Primary and dual grid: Maxwell's paper on the geometrical
  association of physical quantities \cite{maxwell_mathematical}: all
  on the primary grid leads to numerical losses (non physical)
\item
  Maxwell's equations are topological relations about geometrical
  quantities: what makes different schemes different are the
  ``Constitutive relations'' --> Tonti
\item
  Stability problem: Maxwell's equation as a ``control problem'': all
  the eigvals must lie in precise positions --> Taflove
\end{itemize}

for each scheme: constitutive laws, pros, cons.

\section{Structured}
\begin{itemize}
\item
  easy to build
\item
  easy to manage
\item
  staircase approximation \cite{cangellaris_analysis}
\item
  anisotropy of the medium ``grid'' \cite{liu_fourier}
\end{itemize}

\subsection{Orthogonal}
diagonal constitutive matrices: local relations, stable

\subsection{Non-Orthogonal}
non diagonal constitutive matrices: non local constitutive relations,
more unstable (depending on the minimum angle). Mainly used in
association with periodic boundary conditions to simulate non
orthogonal primitive cells in periodic structures.




\section{Unstructured}
Numerical Errors of the Stair-Cased approximation
\cite{cangellaris_analysis}

\begin{itemize}
\item
  better discretization of the domain (tangential fields always
  continuous)
\item
  less memory: coarser grid where I don't need it
\item
  more difficult to generate and manage (meshing software survey)
\end{itemize}

Unstructured grids in 3D are difficult: 2.5D! unstructured in 2D,
structured in the third dimension: layered structured
\cite{gedney_parallel}.

Triangles as simplicials is the most general choice. How do I choose
the dual grid?

\subsection{Delaunay-Vorono\"{i}} \label{sec:delaunay_voronoi}
properties and difficulty in generating it (direct analogous to the
orthogonal structured grid, but unstructured): local constitutive
relations

\subsection{Poincar\'e}
easier to build, but non local relations (non diagonal stiffness matrix)

\subsection{Barycentric}
the more general and stable scheme, but much more complex to implement
(ref. marrone, tonti, bossavit)




\section{Structured vs. Unstructured aka Refractive Index Smoothing}
