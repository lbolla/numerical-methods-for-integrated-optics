%% packages	
\usepackage{bbm,latexsym,afterpage,amsmath,amsfonts}
\usepackage[mathcal]{euscript}
\DeclareMathAlphabet{\mathpzc}{OT1}{pzc}{m}{it}
\DeclareMathOperator{\Area}{Area}
\DeclareMathOperator{\diag}{diag}
\DeclareMathOperator{\Toeplitz}{Toeplitz}

\def \Flush {\afterpage{\clearpage}}

%% generic vector
\newcommand{\Vector}[1]{\overrightarrow{#1}}
%%\newcommand{\Vector}[1]{\vec{#1}}
%%\newcommand{\Vector}[1]{\overline{#1}}

%% vector in time domain
\newcommand{\VectorT}[1]{\widetilde{\Vector{#1}}}
%%\newcommand{\VectorT}[1]{\tilde{\Vector{#1}}}
%%\newcommand{\VectorT}[1]{\underrightarrow{#1}}

\newcommand{\Versor}[1]{\widehat{#1}}
\newcommand{\Array}[1]{\mathbf{#1}}
\newcommand{\Matrix}[1]{\mathbf{#1}}
\newcommand{\Tensor}[1]{\underline{#1}}
\newcommand{\Operator}[2]{\mathbf{#1}\left[#2\right]}
\newcommand{\Set}[1]{\mathcal{#1}}
\def \NaturalSet {\mathbb{N}}
\def \AllNaturalSet {\mathbb{Z}}
\def \RationalSet {\mathbb{Q}}
\def \RealSet {\mathbb{R}}
\def \ImagSet {\mathbb{I}}
\def \ComplexSet {\mathbb{C}}
\def \EmptySet {\left\{ \right\}}
\def \eqdef {\triangleq}

%% mathematical functions
\newcommand{\Conj}[1]{{{#1}^*}}
\newcommand{\Abs}[1]{{\left\lvert#1\right\rvert}}
\newcommand{\Det}[1]{{\left\lvert#1\right\rvert}}
\newcommand{\Arg}[1]{\angle#1}
\newcommand{\Norm}[1]{\left\lVert#1\right\rVert}
\newcommand{\NormOne}[1]{{\Norm{#1}}_1}
\newcommand{\NormTwo}[1]{{\Norm{#1}}_2}
\newcommand{\NormInf}[1]{{\Norm{#1}}_\infty}
\newcommand{\Prod}[2]{{#1\,#2}} % for matrix product
\newcommand{\DotProd}[2]{{#1 \cdot #2}}
\newcommand{\CrossProd}[2]{{#1 \times #2}}
\newcommand{\Real}[1]{{\mathbb{R}\left[#1\right]}}
\newcommand{\Imag}[1]{{\mathbb{I}\left[#1\right]}}
\newcommand{\Transpose}[1]{{#1}^T}
\newcommand{\Hermitian}[1]{{#1}^*}
\newcommand{\Fourier}[1]{\mathcal{F}\left[#1\right]}
\newcommand{\BigO}[1]{\mathcal{O}\left[#1\right]}

\newcommand{\Disc}[3]{{\ }^{#3}{#1}^{#2}}

\newcommand{\Map}[1]{\stackrel{#1}{\longmapsto}}

%% differential operators
\def \d {d}
\def \dt {\d_t}
\def \dx {\d_x}
\def \dy {\d_y}
\def \dz {\d_z}
\def \pd {\partial}
\def \pdt {\pd_t}
\def \pdx {\pd_x}
\def \pdy {\pd_y}
\def \pdz {\pd_z}
\def \GradOperator {\nabla}
\newcommand{\Grad}[1]{{\GradOperator #1}}
\newcommand{\Rot}[1]{\CrossProd{\GradOperator}{#1}}
\newcommand{\Div}[1]{\DotProd{\GradOperator}{#1}}
\newcommand{\Lap}[1]{{\GradOperator^2 #1}}
\newcommand{\Lapt}[1]{{\GradOperator^2_T #1}}
%%\newcommand{\Lap}[1]{{\Delta #1}}
\newcommand{\Bra}[1]{\langle\left.{#1}\right|}
\newcommand{\Ket}[1]{\left|{#1}\right.\rangle}
\newcommand{\BraKet}[2]{\langle{#1}|{#2}\rangle}
\newcommand{\Evaluate}[2]{#1\lvert_{#2}}

%% geometrical elements
\newcommand{\Instant}[1]{\mathpzc{#1}}
\newcommand{\Interval}[1]{\mathpzc{#1}}
\newcommand{\Point}[1]{\mathpzc{#1}}
\newcommand{\Line}[1]{\mathpzc{#1}}
\newcommand{\Surface}[1]{\mathpzc{#1}}
\newcommand{\Volume}[1]{\mathpzc{#1}}
\newcommand{\Dual}[1]{\widetilde{#1}}
\newcommand{\DualInstant}[1]{\Dual{\mathpzc{#1}}}
\newcommand{\DualInterval}[1]{\Dual{\mathpzc{#1}}}
\newcommand{\DualPoint}[1]{\Dual{\Point{#1}}}
\newcommand{\DualLine}[1]{\Dual{\Line{#1}}}
\newcommand{\DualSurface}[1]{\Dual{\Surface{#1}}}
\newcommand{\DualVolume}[1]{\Dual{\Volume{#1}}}
\newcommand{\ExtOrient}[1]{\Dual{#1}}
\newcommand{\Measure}[1]{\Abs{#1}}

\def \nn {n_{\Point{n}}}
\def \ne {n_{\Line{e}}}
\def \nf {n_{\Surface{f}}}
\def \nv {n_{\Volume{v}}}
\def \ndn {n_{\DualPoint{n}}}
\def \nde {n_{\DualLine{e}}}
\def \ndf {n_{\DualSurface{f}}}
\def \ndv {n_{\DualVolume{v}}}

%% special text
\newcommand{\Code}[1]{\texttt{#1}}
\newcommand{\Cite}[2]{\begin{flushright}\emph{#1}\\#2\end{flushright}}
\newcommand{\Sign}[2]{\vspace{1cm}\begin{flushright}#1\\#2\end{flushright}}
\newcommand{\OKKIO}[1]{\textbf{OKKIO: #1}}
\def \FDTD {\Code{FDTD} }
\def \FEM {\Code{FEM} }
\def \threeDFDTD {\Code{3D-FDTD} }
\def \PWE {\Code{PWE} }
\def \DFT {\Code{DFT} }
\def \FFT {\Code{FFT} }
\def \IFFT {\Code{IFFT} }

%% constants
%%\def \CVD {\begin{flushright}$\Box$\end{flushright}}
\def \CVD {$\square$}
\def \E {\Vector{E}}
\def \D {\Vector{D}}
\def \B {\Vector{B}}
\def \H {\Vector{H}}
\def \J {\Vector{J}}
\def \M {\Vector{M}}
\def \k {\Vector{k}}
\def \r {\Vector{r}}
\def \eps {\varepsilon}
\def \deltat {\Delta t}
\def \deltax {\Delta x}
\def \deltay {\Delta y}
\def \deltaz {\Delta z}

\def \Anything {\bullet}
\def \degrees {^\circ}

\newtheorem{theorem}{Theorem}
\newtheorem{proof}{Proof}
\newtheorem{definition}{Definition}

%% for the index 
\newcommand{\tab}[1]{{\it #1}}
\newcommand{\fig}[1]{{\bf #1}}

\newcommand{\figref}[1]{Figure \ref{#1}}
\newcommand{\tabref}[1]{Table \ref{#1}}
\newcommand{\parref}[1]{Part \ref{#1}}
\newcommand{\charef}[1]{Chapter \ref{#1}}
\newcommand{\secref}[1]{Section \ref{#1}}
\newcommand{\appref}[1]{Appendix \ref{#1}}
\newcommand{\defref}[1]{Definition \ref{#1}}
\newcommand{\diaref}[1]{Diagram \ref{#1}}

%% hyphenation 
\hyphenation{Max-well eigen-space smoo-th-ly}
