\part{Mode Solvers} \label{par:mode_solvers} \index{Mode!solvers|(}

\section*{What are the ``Mode Solvers''?}

One of the most powerful approaches to study optical systems is the
eigenmode decomposition: the electromagnetic propagation is ex\-press\-ed
as a set of definite-frequency time-harmonic modes. In the absence of
nonlinear effects, all the optical phenomena can be described as a
super-imposition of these modes with an accuracy directly related to
the number of modes used in the expansion and the completeness of the
set of modes, seen as a basis functions set. All the possible
algorithm formulations to find modes share a common characteristic:
they reduce the problem to find the eigenvalues and eigenvectors of
a given matrix, which represents the geometry and physics of the
device under study.

These algorithms are what we call ``Mode Solvers''\index{Mode!solvers}.

In the next chapters, we will describe two particular algorithms: the
first, suitable to study waveguides with an invariant cross section
along the direction of propagation, based on the Finite Difference
Method, the second, suitable to study periodic structures, also known
as photonic crystal, based on the Plane Wave Expansion technique. They
somehow complete themselves, covering the greatest majority of the
common problems found in optical systems.

\input finite_difference_method

\input plane_wave_expansion_technique

\index{Mode!solvers|)}