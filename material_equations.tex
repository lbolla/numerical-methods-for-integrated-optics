\chapter{Material equations} \label{cha:material_equations}

\section{Introduction}

In \secref{sec:geometry_and_physics}, we have described how to
associate physical quantities to the geometrical elements of the
computational mesh. We have also pointed out that Maxwell equations are
purely topological relations, exactly satisfied by the degrees of
freedom chosen, while errors, unavoidably introduced by the
discretization process, are caused only by the non-perfect modeling of
material equations. The resulting material matrices, also called \emph{mass
matrices} in Finite Elements context and \emph{discrete Hodge
operators} in differential geometry, greatly affects the stability and
the accuracy of time- and frequency-domain algorithms
\cite{schuhmann_whitney,schuhmann_stability}.

In the next sections we'll describe how to build the material
matrices, given a primal mesh, for different dual meshes: in
particular, for the Vorono\"i\index{Vorono\"i}, Poincar\'e\index{Poincar\'e} and barycentric\index{Barycentric!dual mesh} dual grids.

\section{Vorono\"i Dual Mesh} \index{Vorono\"i|(} \label{sec:voronoi}

As described in \secref{sec:dual_mesh}, given a primal simplicial Delaunay\index{Delaunay}
mesh, its Vorono\"i dual mesh is built connecting the spherocenters of
the primal cells. Some properties hold:
\begin{itemize}
\item
  dual edges are orthogonal to primal faces;
\item
  primal edges are orthogonal to dual faces;
\item
  dual volumes are the loci of nearest points to the corresponding
  primal point.
\end{itemize}

These properties represent, as we'll see later, an advantage in modeling
Maxwell equations, but they represent tough geometrical
constrains that meshing software must satisfy in order to generate
proper meshes \cite{triangle}.

Consider a Delaunay primal mesh and its Vorono\"i dual mesh and let:
\begin{equation} \label{eqn:voronoi_dof} \begin{aligned}
e_i & = \DotProd{\E}{\Versor{p}_i} \Measure{\Line{e}_i} \\
h_i & = \DotProd{\H}{\Versor{s}_i} \Measure{\DualLine{e}_i} \\
b_i & = \DotProd{\B}{\Versor{s}_i} \Measure{\Surface{f}_i} \\
d_i & = \DotProd{\D}{\Versor{p}_i} \Measure{\DualSurface{f}_i},
\end{aligned} \end{equation}
where $\Versor{p}_i$ ($\Versor{s}_i$) is the primal (dual) edge
versor, be the realization of the mapping discussed in
\eqref{eqn:dof}. The integral is not necessary because edges are
considered straight and $\E$ is considered piecewise constant along each
edge. From the definition of Vorono\"i dual mesh, $\Versor{p}_i$ is
orthogonal to $\Versor{s}_i$. See \figref{fig:voronoi}.

\begin{figure}[htbp]
  \begin{center}
    \subfigure[The dual edge $\DualLine{e}_i$ is orthogonal to
      the primal face $\Surface{f}_i$.]{\resizebox{!}{5cm}{\input{pics/voronoi_primal.pdf_t}}}
    \subfigure[The edge $\Line{e}_i$ is orthogonal to
      the dual face $\DualSurface{f}_i$]{\resizebox{!}{5cm}{\input{pics/voronoi_dual.pdf_t}}}
  \end{center}
  \caption{Vorono\"i dual mesh.}
  \label{fig:voronoi}
\end{figure}

Maxwell equations read:
\begin{equation} \label{eqn:voronoi_maxwell} \begin{cases}
    \Disc{b_i}{}{n+1/2} & = \Disc{b_i}{}{n-1/2} - \deltat \sum_{j \in
    \Set{B}_{\Surface{f}_i}} \Disc{e_j}{}{n} \\
    \Disc{d_i}{}{n+1} & = \Disc{d_i}{}{n-1} + \deltat \sum_{j \in
    \Set{B}_{\DualSurface{f}_i}} \Disc{h_j}{}{n+1/2}.
  \end{cases} \end{equation}

$\Set{B}_{\Surface{f}_i}$ is the set of the indices that identifies the border
edges of the primal face $\Surface{f}_i$:
\begin{equation*}
  \Set{B}_{\Surface{f}_i} = \left\{ j\ :\ \Line{e}_j \in \partial \Surface{f}_i \right\}
\end{equation*}
where $\partial \Surface{f}$ denotes the boundary of the face $\Surface{f}$, and
$\Set{B}_{\DualSurface{f}_i}$ is the set of the indices that identifies the border
edges of the dual face $\DualSurface{f}_i$:
\begin{equation*}
  \Set{B}_{\DualSurface{f}_i} = \left\{ j\ :\ \DualLine{e}_j \in \partial \DualSurface{f}_i \right\}
\end{equation*}

Thank to the properties of Vorono\"i meshes, the constitutive
equations are straightforward. From \eqref{eqn:voronoi_dof}, consider
the $i^{th}$ primal face, and the corresponding $i^{th}$ (orthogonal) dual edge; we
can write:
\begin{equation} \label{eqn:voronoi_constitutive_magnetic}
  h_i = \frac{\Measure{\DualLine{e}_i}}{\Measure{\Surface{f}_i} \mu_i} b_i
\end{equation}
where $\mu_i$ is the magnetic permeability associated to the primal
face $\Surface{f}_i$: it is supposed piecewise constant on
it. $\Measure{\DualLine{e}}$ and $\Measure{\DualSurface{f}}$ are the
measures of the dual edge $\DualLine{e}$ and the dual face
$\DualSurface{f}$, respectively; as we can note, material matrices
require a metric to be defined \cite{bossavit_computational}: they are
not purely topological relations! We have chosen to use the classical
Euclidean norm, as measure.

The other constitutive equation, for the $i^{th}$ primal edge and the
$i^{th}$ dual face, reads:
\begin{equation} \label{eqn:voronoi_constitutive_electric}
  e_i = \frac{\Measure{\Line{e}_i}}{\Measure{\DualSurface{f}_i} \epsilon_i} d_i
\end{equation}
where $\epsilon_i$ is the electric permeability associated to the dual
face $\DualSurface{f}_i$: it is supposed piecewise constant on it.

Note that the value of $h_i$ ($e_i$) only depends on the value of
$b_i$ ($d_i$) on the corresponding (dual) face. Rewriting
\eqref{eqn:voronoi_constitutive_magnetic} and
\eqref{eqn:voronoi_constitutive_electric} in matrix form, we have:
\begin{align*}
  \Array{h} & = \Prod{\Matrix{M}_\mu}{\Array{b}} \\
  \Array{e} & = \Prod{\Matrix{M}_\epsilon}{\Array{d}} ,
\end{align*}
where $\Array{h}$, $\Array{b}$, $\Array{e}$ and $\Array{d}$ are the arrays of all the $h_i$,
$b_i$, $e_i$ and $d_i$ in \eqref{eqn:voronoi_maxwell} and $\Matrix{M}_\epsilon$ and
$\Matrix{M}_\mu$ are square diagonal matrices whose dimensions are the number
of dual and primal faces, respectively:
\begin{align*}
  \Matrix{M}_\mu \in \nf \times \nf && \Matrix{M}_\epsilon \in \ndf \times \ndf .
\end{align*}

\index{Vorono\"i|)}

\section{Poincar\'e Dual Mesh} \index{Poincar\'e|(} \label{sec:poicare}

Given a primal triangular mesh, the Poincar\'e dual mesh is built
connecting the barycenters of the primal volumes. Comparing to the
Vorono\"i dual mesh, we can note:
\begin{enumerate}
\item
  the dual points are always inside the primal volumes: this makes the
  scheme more stable in the time-domain;
\item
  the dual edges are not orthogonal to the primal faces and the primal
  edges are not orthogonal to the dual faces: this makes the
  constitutive relations more complicated to implement.
\end{enumerate}

Consider a 2D triangular primal mesh, as in \figref{fig:poincare} and
let's limit to the study of the TE polarization\footnote{for the TM
polarization, the Duality Theorem can be applied
\cite{someda_electromagnetic}; the extension to the 3D case is
straightforward, too.}: let $E_z$ and $D_z$ be associated to the nodes
of the primal grid and $\H_t$ and $\B_t$ to the edges of the dual grid.

\begin{figure}[htbp]
  \begin{center}
    \resizebox{4cm}{!}{\input{pics/poincare.pdf_t}}
  \end{center}
  \caption{Poincar\'e dual mesh: primal (dual) edges are not
    orthogonal to dual (primal) faces. In red, identified with $j$,
    are the primal faces (actually edges in the two-dimensional mesh)
    sharing a common vertical primal edge (actually a point) with the
    primal face (actually an edge) $i$.}
  \label{fig:poincare}
\end{figure}

Let:
\begin{equation} \begin{aligned} \label{eqn:poincare_dof}
e_i & = \DotProd{\E}{\Versor{p}_i} \Measure{\Line{e}} \\
h_i & = \DotProd{\H}{\Versor{s}_i} \Measure{\DualLine{e}_i} \\
b_i & = \DotProd{\B}{\Versor{n}_{p_i}} \Measure{\Surface{f}_i} \\
d_i & = \DotProd{\D}{\Versor{n}_{s_i}} \Measure{\DualSurface{f}_i}
\end{aligned} \end{equation}
where $\Versor{p}_i$ ($\Versor{s}_i$) is the primal (dual) edge
versor and $\Versor{n}_{p_i}$ ($\Versor{n}_{s_i}$) is the versor
orthogonal to the primal (dual) face. As noted before,
$\Versor{n}_{p_i}$ is not parallel to $\Versor{s}_i$ and
$\Versor{n}_{s_i}$ is not parallel to $\Versor{p}_i$ as in
Vorono\"i dual grids.

Maxwell equations, being topological relations, are the same as for
the Vorono\"i dual mesh, in \eqref{eqn:voronoi_maxwell}.

Constitutive equations are more complicated because $\H_t \nparallel
\B_t$. The electric constitutive equation
\eqref{eqn:voronoi_constitutive_electric}, valid in the Vorono\"i
case, holds also in this case: this is because we are dealing now only
with the TE case, for which the electric field component $E_z$ is
orthogonal to dual faces (which lie on the $xy$ plane). For the
magnetic constitutive equation, recall that we have to write:
\begin{equation*}
  \Array{h} = \Prod{\Matrix{M}_\mu}{\Array{b}} ,
\end{equation*}
where $\Array{h}$ and $\Array{b}$ are the arrays of all the $h_i$ and
$b_i$ in \eqref{eqn:voronoi_maxwell} and $\Matrix{M}_\mu$ is a square
matrix whose dimension is the number of dual edges (or primal faces).

This can be done in three steps \cite{taflove_advances}.
\begin{enumerate}
\item
  Consider the primal face $\Surface{f}_i$ and all the faces
  $\Surface{f}_j$ sharing with it one edge, like in
  \figref{fig:poincare}. Write $\B_j$, the magnetic field on the face
  $\Surface{f}_j$, as a linear combination of $b_i$ and $b_j$,
  magnetic fluxes through $\Surface{f}_i$ and $\Surface{f}_j$:
  \begin{equation*} \label{eqn:poincare_first_step} \begin{cases}
      \DotProd{\B_j}{\Vector{N}_{p_i}} =
      \DotProd{\B}{\Vector{N}_{p_i}} = b_i \\
      \DotProd{\B_j}{\Vector{N}_{p_j}} =
      \DotProd{\B}{\Vector{N}_{p_j}} = b_j .
  \end{cases} \end{equation*}
  where $\Vector{N}_{p_i} = \Versor{n}_{p_i} \Measure{\Surface{f}_i}$
  and the same for $\Vector{N}_{p_j}$. These relations come from the
  imposition of flux conservation.
  
  The right-hand sides in \eqref{eqn:poincare_first_step} are known
  from Faraday equation (the first in \eqref{eqn:voronoi_maxwell}), so
  we can calculate $\B_j$:
  \begin{align*}
    \B_j = \Prod{\Matrix{N}^{-1}}{\begin{bmatrix} b_i \\ b_j \end{bmatrix}}
    && \text{where} && \Matrix{N} = \begin{bmatrix} N^x_{p_i}
    & N^y_{p_i} \\ N^x_{p_j} & N^y_{p_j} \end{bmatrix} .
  \end{align*}
\item
  Write $\B$ as a linear combination of $\B_j$:
  \begin{equation*}
    \B_i = \frac{1}{W} \sum_j w_j \B_j ,
  \end{equation*}
  where $W = \sum_j w_j$ and $w_j = \Norm{\DotProd{\Versor{z}}{\left(
  \CrossProd{\Vector{N}_{p_i}}{\Vector{N}_{p_j}} \right)}}$ are
  weighting coefficients whose value is twice the area of the triangle
  identified by the edges $i$ and $j$.
\item
  Finally, compute $\H$ from the continuum constitutive relation $\H = \mu^{-1} \B$ and
  then $h_i$, from \eqref{eqn:poincare_dof}:
  \begin{equation} \label{eqn:poincare_constitutive_magnetic} \begin{split}
    h_i & = \DotProd{\H}{\Versor{s}_i} \Measure{\DualLine{e}_i} \\
    & = H_x \Measure{\DualLine{e}_i} \cos\beta + H_y \Measure{\DualLine{e}_i} \sin\beta \\
    & = \mu_x^{-1} B_x \Measure{\DualLine{e}_i} \cos\beta +
    \mu_y^{-1} B_y \Measure{\DualLine{e}_i} \sin\beta \\
    & = \frac{1}{W} \sum_j w_j \left[ \mu_x^{-1} B_x \Measure{\DualLine{e}_i}
    \cos\beta + \mu_y^{-1} B_y \Measure{\DualLine{e}_i}
    \sin\beta \right] \\
    & = \frac{1}{W} \sum_j w_j \left[ \begin{split} \mu_x^{-1} \Measure{\DualLine{e}_i}
    \cos\beta \left( k_{11} b_i + k_{12} b_j \right) \\
    + \mu_y^{-1} \Measure{\DualLine{e}_i} \sin\beta \left( k_{21} b_i +
    k_{22} b_j \right) \end{split} \right] \\
    & = \frac{\Measure{\DualLine{e}_i}}{W} \sum_j w_j \left[
    \begin{split} \left( \mu_x^{-1} k_{11}
    \cos\beta + \mu_y^{-1} k_{21} \sin\beta
    \right) b_i \\ + \left( \mu_y^{-1} k_{12}
    \cos\beta + \mu_y^{-1} k_{22} \sin\beta
    \right) b_j \end{split} \right]
  \end{split} \end{equation}
  where $\beta$ is the angle between the dual edge $\DualLine{e}_i$ and
  the $x$-axis and $\Matrix{K} = \bigl[ \begin{smallmatrix}
  k_{11}&k_{12}\\k_{21}&k_{22} \end{smallmatrix} \bigr] =
  \Matrix{N}^{-1}$.
  
  In matricial form, \eqref{eqn:poincare_constitutive_magnetic} reads:
  \begin{equation*}
    \Array{h} = \Prod{\Matrix{M}_\mu}{\Array{b}} ,
  \end{equation*}
  where
  \begin{equation} \label{eqn:constitutive_poincare}
    \Matrix{M}_\mu = \begin{bmatrix}
      {\Matrix{M}_\mu}_{ij}
    \end{bmatrix} = \begin{cases}
      \frac{\Measure{\DualLine{e}}}{W} \sum_j w_j 
    \left( \mu_x^{-1} k_{11} \cos\beta + \mu_y^{-1} k_{21} \sin\beta
    \right) & \text{for $i = j$} \\
      \frac{\Measure{\DualLine{e}}}{W} w_j 
    \left( \mu_x^{-1} k_{12} \cos\beta + \mu_y^{-1} k_{22} \sin\beta
    \right) & \text{for $i \neq j$} \\
    \end{cases}
  \end{equation}
\end{enumerate}

It is worth noting that:
\begin{itemize}
\item
  these three steps are fully explicit, i.e. we
  don't ever need to solve a linear system to compute the unknowns we
  need;
\item
  the final matrix $\Matrix{M}_\mu$ is not diagonal, as
  explicitly stated in \eqref{eqn:constitutive_poincare}: this is not
  a problem, neither in the time- nor in the frequency-domain;
\item
  \eqref{eqn:poincare_constitutive_magnetic} is valid also for
  anisotropic materials: just use a tensor $\mu = \left[
  \begin{smallmatrix} \mu_{xx} & \mu_{xy} \\ \mu_{yx} & \mu_{yy}
  \end{smallmatrix} \right]$. Note also that dealing with an
  anisotropic material is equivalent in introducing a local change of
  metric. In fact, for the simple case of a coordinate system oriented
  with the crystallographic axis of the anisotropic
  material\footnote{If not, a simple rotation of the reference axis
  can bring us to this condition}:
  \begin{equation*} \begin{split}
    h_i & = \Prod{\left( \mu^{-1} \B \right)}{\Versor{s}_i} \Measure{\DualLine{e}_i} \\
        & = \left( \mu_x^{-1} {s_i}_x B_x + \mu_y^{-1} {s_i}_y B_y \right) \Measure{\DualLine{e}_i} \\
        & = \Prod{\B}{\Vector{s}'_i} \Measure{\DualLine{e}_i}
  \end{split} \end{equation*}
  where $\Vector{s}'_i = \left( \mu_x^{-1} {s_i}_x, \mu_y^{-1}
  {s_i}_y \right)$. The metric, as said before, is needed when dealing
  with material equations.
\end{itemize}

\index{Poincar\'e|)}

\section{Barycentric Dual Mesh} \index{Barycentric!dual mesh|(} \label{sec:barycentric}

Given a triangular primal mesh, its barycentric dual is made of:
\begin{itemize}
\item
  dual points that are the barycenters of the primal volumes;
\item
  dual edges that are piece-lines made by two straight lines whose
  contact point is the barycenter of the corresponding primal face,
  and connect the dual points;
\item
  dual surfaces having as a border the dual lines surrounding the
  corresponding primal edge.
\end{itemize}

\begin{figure}[htbp]
  \begin{center}
    \resizebox{4cm}{!}{\input{pics/barycentric.pdf_t}}
  \end{center}
  \caption{Barycentric dual mesh: dual edges are piece-lines to
  connect barycenters of adjacent primal cells.}
  \label{fig:barycentric}
\end{figure}

Since primal and dual meshes are not orthogonal, as in the Poincar\'e
mesh, the constitutive equation requires a similar mathematical
procedure to be worked out. The advantage of this mesh is its superior
stability and the possibility to use any kind of triangular grid, not
only a Delaunay one \cite{marrone_computational}.

Closely following \cite{marrone_computational}, we have implemented the
material matrices for the two-dimensional mesh and both the
polarizations. It has been used to study dispersive materials, as
described in \secref{sec:nim}

\index{Barycentric!dual mesh|)}

\section{Particular Material Equations} \label{sec:particular_material_equations}

In the previous sections the conventional material equations have been
discretized for different dual meshes. We'll show now how more complex
materials can be modeled with the present method. Another example can be
found in \secref{sec:nim}.

\subsection{Ohm Losses} \label{sec:ohm_losses}

The description of the Ohm losses can be easily included in
\eqref{eqn:discrete_ampere} and \eqref{eqn:discrete_faraday}, as follows:
\begin{equation*} \begin{cases}
    \Prod{\Transpose{\Matrix{R}}}{\Array{h}} = \partial_t \Array{d} + \Array{j} + \Matrix{\Omega_{EN}} \Array{e} \\
    \Prod{\Matrix{R}}{\Array{e}} = -\partial_t \Array{b} + \Array{m} - \Matrix{\Omega_{MN}} \Array{h}
\end{cases} . \end{equation*}

According to the association of the degrees of freedom to the
geometrical elements of the mesh, shown in
\secref{sec:geometry_and_physics}, supposing that given a primal edge,
the electric field is uniform on it and the conductibility is uniform
over the corresponding dual face, we can easily build the matrices
$\Matrix{\Omega_{EN}}$ and $\Matrix{\Omega_{MN}}$ with the procedures
described in the previous sections for the material matrices,
depending on the particular dual grid chosen. The only problem arises
in the time domain: in fact, the matrix/operator $\Matrix{\Omega_{EN}}$
($\Matrix{\Omega_{MN}}$) must link a physical quantity defined on the
primal instants (the electric field) with one defined on the dual
instants (the magnetic field). Hence, it must be estimated by some
means of time approximation. The easiest way to do it, which keeps the
algorithm stable, is to approximate the electric field at the
time-step $n+1/2$ as the mean value of the instants before and after
it:
\begin{equation*}
  \Disc{\Array{e}}{}{n+1/2} \simeq \frac{\Disc{\Array{e}}{}{n} +
  \Disc{\Array{e}}{}{n+1}}{2} .
\end{equation*}

The same reasoning holds for the magnetic conductivity matrix
$\Matrix{\Omega_{MN}}$\footnote{Even if the magnetic conductivity can
  appear a useless physical quantity, it is not, because it is used in
  the modelling of PML losses}.

It should be clear that this problem only arises in the
time-domain. In the frequency domain, the easiest way to deal with Ohm
losses is to accept a complex value for the electric and magnetic
permeabilities. In formulas:
\begin{equation*}
    \begin{cases}
      \Rot{\H} = -\imath \omega \epsilon \E + \sigma_e \E \\
      \Rot{\E} = \imath \omega \mu \H + \sigma_m \H \\
    \end{cases}
    \Longrightarrow
    \begin{cases}
      \Rot{\H} = -\imath \omega \bar{\epsilon} \E \\
      \Rot{\E} = \imath \omega \bar{\mu} \H \\
    \end{cases}
\end{equation*}
where $\bar{\epsilon} = \epsilon + \imath \frac{\sigma_e}{\omega}$ and
$\bar{\mu} = \mu + \imath \frac{\sigma_m}{\omega}$. Obviously, also
the material matrices $\Matrix{M}_\epsilon$ and $\Matrix{M}_\mu$ will
be complex.

For later convenience, let $\Matrix{P}_m =
\Prod{\Matrix{M}_\mu}{\Matrix{\Omega}_{MN}}$ and $\Matrix{P}_e =
\Prod{\Matrix{M}_\epsilon}{\Matrix{\Omega}_{EN}}$.

\subsection{PML} \label{sec:pml}

PMLs are used to model materials with infinite extent, i.e. to model
the radiation of light as if the boundaries of the computational
domain were at infinite distance from the source. There are mainly two
formulations to model the PMLs:
\begin{description}
\item[coordinate stretching]: in which a local change of metric is
  introduced where attenuation is needed, so that any plane wave
  impinging at the interface undergoes no reflection, but
  exponentially decays at zero as propagating inside the PMLs;
\item[Uniaxial PML]: in which an anisotropic lossy material, but fully
  Maxwellian, is defined, with the property of absorbing, without
  reflections, any plane wave impinging to it.
\end{description}

Even if both the formulations have been successfully implemented in
many algorithms, we have decided to implement the second. The main
reason is that the U-PML are a fully Maxwellian medium, while the
medium coming out from the coordinate stretching is not: Maxwell
equations hold in the U-PMLs, but not in the coordinate stretched
system and new equation must be implemented\footnote{In particular,
Gauss laws fail to hold: at the interface between two different
media inside the PMLs, the normal components of both $\E$ and $\D$ are
continuous, instead of just $\D$. The same is valid for $\H$ and
$\B$.}. Having a scheme of degrees of freedom and geometrical elements
fully functional to solve Maxwell equations it seems a waste of time
to introduce a new one only where PMLs are needed: sometimes, laziness
is a good quality for scientists and programmers.

Uniaxial PMLs are an anisotropic lossy medium. Maxwell equations in
this medium read:
\begin{equation*} \begin{cases}
    \Rot{\E} = -\imath \omega \Tensor{\mu} \H \\
    \Rot{\H} = \imath \omega \Tensor{\epsilon} \E + \J
\end{cases} , \end{equation*}
where $\Tensor{\mu}$ and $\Tensor{\epsilon}$ are tensors defined as:
\begin{align*}
  \Tensor{\mu} & = \Prod{\mu}{\Tensor{s}} \\
  \Tensor{\epsilon} & = \Prod{\epsilon}{\Tensor{s}}
\intertext{and:}
  \Tensor{s} & = \Prod{\Tensor{s}_x}{\Prod{\Tensor{s}_y}{\Tensor{s}_z}} \\
  \Tensor{s}_x & = \begin{bmatrix} s_x^{-1} & 0 & 0 \\ 0 & s_x & 0 \\ 0 & 0 & s_x \end{bmatrix} \\
  \Tensor{s}_y & = \begin{bmatrix} s_y & 0 & 0 \\ 0 & s_y^{-1} & 0 \\ 0 & 0 & s_y \end{bmatrix} \\
  \Tensor{s}_z & = \begin{bmatrix} s_z & 0 & 0 \\ 0 & s_z & 0 \\ 0 & 0 & s_z^{-1} \end{bmatrix}
\intertext{and, finally:}
  s_x & = k_x + \frac{\sigma_x}{\imath \omega \epsilon_0} \\
  s_y & = k_y + \frac{\sigma_y}{\imath \omega \epsilon_0} \\
  s_z & = k_z + \frac{\sigma_z}{\imath \omega \epsilon_0} .
\end{align*}

In the previous sections, we have shown how to describe both
anisotropic and lossy materials, so we have all we need to describe
U-PMLs, at least in the frequency-domain.

In the time-domain, the presence of the frequency $\omega$ in the
definition of the absorbing coefficients introduces a
\emph{non-locality} in time. Material equations in the
frequency-domain, are valid in the form:
\begin{align*}
  \D(\omega) & = \epsilon(\omega) \E(\omega) \\
  \B(\omega) & = \mu(\omega) \H(\omega)
\end{align*}
that, in the time-domain, become a convolution:
\begin{align*}
  \D(t) = \int \epsilon(t - \tau) \E(\tau) \d \tau \\
  \B(t) = \int \mu(t - \tau) \H(\tau) \d \tau
\end{align*}

The consequence is that the value of $\D$ at the time $t$ depends on
the values of $\E$ for all the previous\footnote{Only the previous times,
  not the future, because causality always holds.} times $\tau$.

Using some notation that will be explained in
\charef{cha:time_domain}, we can write Maxwell equations in the
time-domain, which hold in the PMLs:

\begin{equation} \label{eqn:maxwell_time_domain_pml} \left\{ \begin{aligned} 
  \Disc{\Array{d}}{}{n+1} & = \Disc{\Array{d}}{}{n} + \deltat
  \Prod{\Transpose{\Matrix{R}}}{\Disc{\Array{h}}{}{n+1/2}} -
  \deltat \Prod{\Matrix{\Omega}_{EN}}{\Disc{\Array{e}}{}{n+1/2}} -
  \Disc{\Array{j}}{}{n+1/2} \\
  \Disc{\Array{e}}{}{n+1} & = \Disc{\Array{e}}{}{n} +
  \Prod{\Matrix{M}_\epsilon}{\left(\Disc{\Array{d}}{}{n+1} -
  \Disc{\Array{d}}{}{n}\right)} - \deltat
  \Prod{\Matrix{\Omega}_{EC}}{\Disc{\Array{e}}{}{n+1/2}} \\
  \Disc{\Array{b}}{}{n+3/2} & = \Disc{\Array{b}}{}{n+1/2} - \deltat
  \Prod{\Matrix{R}}{\Disc{\Array{e}}{}{n+1}} - \deltat
  \Prod{\Matrix{\Omega}_{MN}}{\Disc{\Array{h}}{}{n+1}} + \Disc{\Array{m}}{}{n+1} \\
  \Disc{\Array{h}}{}{n+3/2} & = \Disc{\Array{h}}{}{n+1/2} +
  \Prod{\Matrix{M}_\mu}{\left(\Disc{\Array{b}}{}{n+3/2} -
  \Disc{\Array{b}}{}{n+1/2} \right)} - \deltat
  \Prod{\Matrix{\Omega}_{MC}}{\Disc{\Array{h}}{}{n+1}}
\end{aligned} \right. \end{equation}
where:
\begin{align*}
  \Matrix{\Omega}_{MN} & = \sigma \frac{\Measure{\Surface{f}}}{\Measure{\DualLine{e}}} &&
  \text{Magnetic Ohm losses} \\
  \Matrix{\Omega}_{MC} & = \frac{\sigma}{\mu_0}  &&
  \text{Magnetic PML losses} \\
  \Matrix{\Omega}_{EN} & = \sigma \frac{\Measure{\DualSurface{f}}}{\Measure{\Line{e}}}  &&
  \text{Electric Ohm losses} \\
  \Matrix{\Omega}_{EC} & = \frac{\sigma}{\epsilon_0} &&
  \text{Electric PML losses} 
\end{align*}

Note that where $\sigma = 0$, \eqref{eqn:maxwell_time_domain_pml}
becomes \eqref{eqn:maxwell_time_domain}: we don't need two different
equations to deal with PMLs.





%% \OKKIO{disegno di un dominio e delle PML: mostrare in che zone
%%   $\sigma = 0$}
